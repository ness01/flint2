% vim: spell spelllang=en textwidth=80
%%%%%%%%%%%%%%%%%%%%%%%%%%%%%%%%%%%%%%%%%%%%%%%%%%%%%%%%%%%%%%%%%%%%%%%%%%%%%%%
%
%   This file is part of FLINT.
%
%   FLINT is free software; you can redistribute it and/or modify
%   it under the terms of the GNU General Public License as published by
%   the Free Software Foundation; either version 2 of the License, or
%   (at your option) any later version.
%
%   FLINT is distributed in the hope that it will be useful,
%   but WITHOUT ANY WARRANTY; without even the implied warranty of
%   MERCHANTABILITY or FITNESS FOR A PARTICULAR PURPOSE.  See the
%   GNU General Public License for more details.
%
%   You should have received a copy of the GNU General Public License
%   along with FLINT; if not, write to the Free Software
%   Foundation, Inc., 51 Franklin St, Fifth Floor, Boston, MA  02110-1301 USA
%
%%%%%%%%%%%%%%%%%%%%%%%%%%%%%%%%%%%%%%%%%%%%%%%%%%%%%%%%%%%%%%%%%%%%%%%%%%%%%%%
%%%%%%%%%%%%%%%%%%%%%%%%%%%%%%%%%%%%%%%%%%%%%%%%%%%%%%%%%%%%%%%%%%%%%%%%%%%%%%%
%
%   Copyright (C) 2007 William Hart, David Harvey
%   Copyright (C) 2010 Sebastian Pancratz
%   Copyright (C) 2013 Tom Bachmann
%
%%%%%%%%%%%%%%%%%%%%%%%%%%%%%%%%%%%%%%%%%%%%%%%%%%%%%%%%%%%%%%%%%%%%%%%%%%%%%%%

\documentclass[a4paper,10pt]{book}

%%%%%%%%%%%%
% geometry %
%%%%%%%%%%%%

\usepackage[hmargin=3.8cm,vmargin=3cm,a4paper,centering,twoside]{geometry}
\setlength{\headheight}{14pt}

% Dutch style of paragraph formatting, i.e. no indents
\setlength{\parskip}{1.3ex plus 0.2ex minus 0.2ex}
\setlength{\parindent}{0pt}

%%%%%%%%%%%%%%%%%%
% Other packages %
%%%%%%%%%%%%%%%%%%

\usepackage{amsmath,amsthm,amscd,amsfonts,amssymb}
\usepackage{cases}
\usepackage[all]{xy}

\usepackage{ifpdf}
\usepackage{paralist}
\usepackage{fancyhdr}
\usepackage{sectsty}
\usepackage{epigraph}
\usepackage{natbib}
\usepackage{url}
\usepackage[T1]{fontenc}
\usepackage{ae,aecompl}
\usepackage{booktabs}
\usepackage{multirow}
\usepackage{verbatim}
\usepackage{listings}

%%%%%%%%%%%%
% hyperref %
%%%%%%%%%%%%

\usepackage{hyperref}
\hypersetup{
    colorlinks=true,    % false: boxed links; true: colored links
    citecolor=green,    % color of links to bibliography
    filecolor=red,      % color of file links
    linkcolor=blue,     % color of internal links
    urlcolor=blue       % color of external links
}

\makeatletter
\newcommand\org@hypertarget{}
\let\org@hypertarget\hypertarget
\renewcommand\hypertarget[2]{%
    \Hy@raisedlink{\org@hypertarget{#1}{}}#2%
} 
\makeatother

\ifpdf
    \hypersetup{
        pdftitle={FLINT},
        pdfauthor={},
        pdfsubject={Computational mathematics},
        bookmarks=true,
        bookmarksnumbered=true,
        unicode=true,
        pdfstartview={FitH},
        pdfpagemode={UseOutlines}
    }
\fi

%%%%%%%%%%
% natbib %
%%%%%%%%%%

\bibpunct{[}{]}{,}{n}{}{}

\renewcommand{\bibname}{References}

%%%%%%%%%%%
% sectsty %
%%%%%%%%%%%

\allsectionsfont{\nohang\centering}

\sectionfont{\nohang\centering\large}

\makeatletter
\renewcommand{\@makechapterhead}[1]{%
\vspace*{50 pt}%
\begin{center}
\bfseries\Huge\S \thechapter.\ #1
\end{center}
\vspace*{40 pt}}
\makeatother

%%%%%%%%%%%%%%%%%%%%%
% Table of contents %
%%%%%%%%%%%%%%%%%%%%%

\usepackage{tocloft}

\addtolength{\cftsecnumwidth}{0.8em}
\addtolength{\cftbeforesecskip}{0.05em}

%%%%%%%%%%%%
% fancyhdr %
%%%%%%%%%%%%

\newcommand\nouppercase[1]{{%
    \let\uppercase\relax
    \let\MakeUppercase\relax
    \expandafter\let\csname MakeUppercase \endcsname\relax#1}%
}

\pagestyle{fancyplain}

\renewcommand{\chaptermark}[1]{\markboth{#1}{}}
\renewcommand{\sectionmark}[1]{\markright{\thesection\ #1}}
\fancyhf{}
\fancyhead[LE,RO]{\bfseries\thepage}
\fancyhead[LO]{\itshape\nouppercase{\rightmark}}
\fancyhead[RE]{\itshape\nouppercase{\leftmark}}
\renewcommand{\headrulewidth}{0pt}
\renewcommand{\footrulewidth}{0pt}

\fancypagestyle{plain}{%
  \fancyhead{}
  \renewcommand{\headrulewidth}{0pt}
}

\makeatletter
\def\cleardoublepage{\clearpage\if@twoside \ifodd\c@page\else
    \hbox{}
    \thispagestyle{plain}
    \newpage
    \if@twocolumn\hbox{}\newpage\fi\fi\fi}
\makeatother \clearpage{\pagestyle{plain}\cleardoublepage}

%%%%%%%
% url %
%%%%%%%

\makeatletter
\def\url@leostyle{%
  \@ifundefined{selectfont}{\def\UrlFont{\sf}}{\def\UrlFont{\small\ttfamily}}}
\makeatother
\urlstyle{leostyle}

%%%%%%%%%%%%%%%%
% Enumerations %
%%%%%%%%%%%%%%%%

\setlength{\pltopsep}{0.24em}
\setlength{\plpartopsep}{0em}
\setlength{\plitemsep}{0.24em}

% This should do what we want
%   \setdefaultenum{(i)}{(a)}{1.}{A}
% but it does not work for references, dropping the 
% parentheses.  The following hack does work.

\renewcommand{\theenumi}{(\roman{enumi})}
\renewcommand{\theenumii}{(\alph{enumii})}
\renewcommand{\theenumiii}{\arabic{enumiii}.}
\renewcommand{\theenumiv}{\Alph{enumiv}}

\renewcommand{\labelenumi}{\theenumi}
\renewcommand{\labelenumii}{\theenumii}
\renewcommand{\labelenumiii}{\theenumiii}
\renewcommand{\labelenumiv}{\theenumiv}

%%%%%%%%%%%%%%%%%%%%%%%%%
% Mathematical commands %
%%%%%%%%%%%%%%%%%%%%%%%%%

\renewcommand{\to}{\rightarrow}%         Right arrow
\newcommand{\into}{\hookrightarrow}%     Injection arrow
\newcommand{\onto}{\twoheadrightarrow}%  Surjection arrow

\providecommand{\abs}[1]{\lvert#1\rvert}%                  Absolute value
\providecommand{\absbig}[1]{\bigl\lvert#1\bigr\rvert}%     Absolute value
\providecommand{\absBig}[1]{\Bigl\lvert#1\Bigr\rvert}%     Absolute value
\providecommand{\absbigg}[1]{\biggl\lvert#1\biggr\rvert}%  Absolute value

\providecommand{\norm}[1]{\lVert#1\rVert}%               Norm
\providecommand{\normbig}[1]{\bigl\lVert#1\bigr\rVert}%  Norm
\providecommand{\normBig}[1]{\Bigl\lVert#1\Bigr\rVert}%  Norm

\providecommand{\floor}[1]{\left\lfloor#1\right\rfloor}%    Floor
\providecommand{\floorbig}[1]{\bigl\lfloor#1\bigr\rfloor}%  Floor
\providecommand{\floorBig}[1]{\Bigl\lfloor#1\Bigr\rfloor}%  Floor

\providecommand{\ceil}[1]{\left\lceil#1\right\rceil}%    Ceiling
\providecommand{\ceilbig}[1]{\bigl\lceil#1\bigr\rceil}%  Ceiling
\providecommand{\ceilBig}[1]{\Bigl\lceil#1\Bigr\rceil}%  Ceiling

\newcommand{\N}{\mathbf{N}}%  Natural numbers
\newcommand{\Z}{\mathbf{Z}}%  Integers
\newcommand{\Q}{\mathbf{Q}}%  Rationals

\DeclareMathOperator{\sgn}{sgn}
\DeclareMathOperator{\ord}{ord}
\DeclareMathOperator{\lcm}{lcm}

\allowdisplaybreaks[4]
%\numberwithin{equation}{section}

%%%%%%%%%%%%
% listings %
%%%%%%%%%%%%

\lstset{language=c}
\lstset{basicstyle=\ttfamily}
\lstset{breaklines=true}
\lstset{breakatwhitespace=true}
\lstset{keywordstyle=}
\lstset{morekeywords={mpz_t, mpq_t, mpz_poly_t, fmpz, fmpz_t, fmpz_poly_t}}
\lstset{escapechar=}
\lstset{showstringspaces=false}

%%%%%%%%%%%%%%%%%%%%%%%%%%%
% FLINT specific commands %
%%%%%%%%%%%%%%%%%%%%%%%%%%%

\newcommand{\code}{\lstinline}

\DeclareMathOperator{\len}{len}

%%%%%%%%%%%%%%%%%%%%%%%%%%%%%%%%%%%%%%%%%%%%%%%%%%%%%%%%%%%%%%%%%%%%%%%%%%%%%%%
% DOCUMENT                                                                    %
%%%%%%%%%%%%%%%%%%%%%%%%%%%%%%%%%%%%%%%%%%%%%%%%%%%%%%%%%%%%%%%%%%%%%%%%%%%%%%%

\begin{document}

%%%%%%%%%%%%%%%%%%%%%%%%%%%%%%%%%%%%%%%%%%%%%%%%%%%%%%%%%%%%%%%%%%%%%%%%%%%%%%%
% FRONTMATTER                                                                 %
%%%%%%%%%%%%%%%%%%%%%%%%%%%%%%%%%%%%%%%%%%%%%%%%%%%%%%%%%%%%%%%%%%%%%%%%%%%%%%%

\frontmatter

\input{input/title.tex}
\clearpage

\tableofcontents

%%%%%%%%%%%%%%%%%%%%%%%%%%%%%%%%%%%%%%%%%%%%%%%%%%%%%%%%%%%%%%%%%%%%%%%%%%%%%%%
% MAINMATTER                                                                  %
%%%%%%%%%%%%%%%%%%%%%%%%%%%%%%%%%%%%%%%%%%%%%%%%%%%%%%%%%%%%%%%%%%%%%%%%%%%%%%%

\mainmatter

\chapter{Introduction}

FLINT is a C library of functions for doing number theory. It is highly 
optimised and can be compiled on numerous platforms.  FLINT also has the 
aim of providing support for multicore and multiprocessor computer 
architectures. To this end, the library is threadsafe, with few exceptions
noted in the appropriate place.

FLINT is currently maintained by William Hart of Warwick University in 
the UK. Its main authors are William Hart, Sebastian Pancratz, Fredrik
Johansson, Andy Novocin and David Harvey (no longer active).

FLINT 2 and following should compile on any machine with GCC and a standard 
GNU toolchain, however it is specially optimised for x86 (32 and 64 bit)
machines. There is also limited optimisation for ARM and ia64 machines. As 
of version 2.0, FLINT required GCC version 2.96 or later, either MPIR
(2.6.0 or later) or GMP (5.1.1 or later), and MPFR 3.0.0 or later.
It is also required that the platform 
provide a \code{uint64_t} type if a native 64 bit type is not available. 
Full C99 compliance is \textbf{not} required.

FLINT is supplied as a set of modules, \code{fmpz}, \code{fmpz_poly}, etc., 
each of which can be linked to a C program making use of their functionality.

All of the functions in FLINT have a corresponding test function provided 
in an appropriately named test file.  For example, the function 
\code{fmpz_poly_add} located in\\ \code{fmpz_poly/add.c} has test code in the 
file \code{fmpz_poly/test/t-add.c}.

\chapter{Building and using FLINT}

The easiest way to use FLINT is to build a shared library.  Simply download 
the FLINT tarball and untar it on your system.

FLINT requires either MPIR (version 2.6.0 or later) or GMP
(version 5.1.1 or later). If MPIR is used, MPIR must be built with
the \code{--enable-gmpcompat} option. FLINT also requires
MPFR 3.0.0 or later and a pthread implementation.
Some of the input/output tests require 
\code{fork} and \code{pipe}, however these are disabled on MinGW which
does not provide a posix implementation.

To configure FLINT you must specify where GMP/MPIR and MPFR are on
your system. FLINT can work with the libraries installed as usual,
e.g. in \code{/usr/local} or it can work with the libraries built
from source in their standard source trees. 

In the case that a library is installed in say \code{/usr}
in the \code{lib} and \code{include} directories as usual, simply
specify the top level location, e.g. \code{/usr} when 
configuring FLINT. If a library is built in its source tree,
specify the top level source directory, e.g. \code{/home/user1/mpir/}.

To specify the directories where the libraries reside, you must
pass the directories as parameters to FLINT's configure, e.g.\ 
\begin{lstlisting}[language=bash]
./configure --with-mpir=/usr --with-mpfr=/home/user1/mpfr/
\end{lstlisting}
If no directories are specified, FLINT assumes it will find the
libraries it needs in \code{/usr/local}.

Note that FLINT builds static and shared libraries by default, except
on platforms where this is not supported. If you do not require either
a shared or static library then you may pass \code{--disable-static} 
or \code{--disable-shared} to \code{configure}. 

If you intend to install the FLINT library and header files, 
you can specify where they should be placed by passing 
\code{--prefix=path} to configure, where \code{path} is the directory
under which the \code{lib} and \code{include} directories exist into
which you wish to place the FLINT files when it is installed.

If you wish to use FLINT on a single core machine then it will be
configured by default for single mode. This is slightly faster, but
is not threadsafe. (This mode can also be explicitly selected by 
passing the \code{--single} option to configure.) If you wish to 
build a threadsafe version of FLINT, you must pass the 
\code{--reentrant} option to configure. This will be slower on 
single core machines, but threadsafe.

On some systems, e.g. Sparc and some Macs, more than one ABI is 
available. FLINT chooses the ABI based on the CPU type available,
however its default choice can be overridden by passing either
\code{ABI=64} or \code{ABI=32} to configure.

In some cases, it is necessary to override the entire CPU/OS
defaults. This can be done by passing \code{--build=cpu-os} to
configure. The available choices for CPU include \code{x86_64},
\code{x86}, \code{ia64}, \code{sparc}, \code{sparc64}, \code{ppc},
\code{ppc64}. Other CPU types are unrecognised and FLINT will
build with generic code on those machines. The choices for OS
include \code{Linux}, \code{MINGW32}, \code{CYGWIN}, 
\code{Darwin}, \code{FreeBSD}, \code{SunOS} and numerous other
operating systems.
 
It is also possible to override the default CC, AR and CFLAGS used
by FLINT by passing \code{CC=full_path_to_compiler}, etc., to 
FLINT's configure.

Once FLINT is configured, in the main directory of the FLINT directory 
tree simply type:
\begin{lstlisting}[language=bash]
make
make check
\end{lstlisting}

GNU make is required to build FLINT. This is simply \code{make} on 
Linux, Darwin, MinGW and Cygwin systems. However, on some unixes the 
command is \code{gmake}.

If you wish to install FLINT, simply type:
\begin{lstlisting}[language=bash]
make install
\end{lstlisting}

Now to use FLINT, simply include the appropriate header files for 
the FLINT modules you wish to use in your C program.  Then compile 
your program, linking against the FLINT library, GMP/MPIR, MPFR and
pthreads with the options \code{-lflint -lmpfr -lgmp -lpthread}.

Note that you may have to set \code{LD_LIBRARY_PATH} or equivalent
for your system to let the linker know where to find these libraries.
Please refer to your system documentation for how to do this.

If you have any difficulties with conflicts with system headers on
your machine, you can do the following in your code:

\begin{lstlisting}[language=C]
#undef ulong
#include <stdio.h>
// other system headers
#define ulong mp_limb_t
\end{lstlisting}

This prevents FLINT's definition of \code{ulong} interfering with
your system headers.

The FLINT make system responds to the standard commands
\begin{lstlisting}[language=bash]
make 
make library
make check
make clean
make distclean
make install
\end{lstlisting}

In addition, if you wish to simply check a single module of FLINT you
can pass the option \code{MOD=modname} to \code{make check}. You can 
also pass a list of module names in inverted commas, e.g:

\begin{lstlisting}[language=bash]
make check MOD=ulong_extras
\end{lstlisting}

If your system supports parallel builds, FLINT will build in parallel,
e.g:
\begin{lstlisting}[language=bash]
make -j4 check 
\end{lstlisting}

Note that on some systems, most notably MinGW, parallel make is 
supported but can be problematic.

\chapter{Test code}

Each module of FLINT has an extensive associated test module.  We 
strongly recommend running the test programs before relying on results 
from FLINT on your system. 

To make and run the test programs, simply type:
\begin{lstlisting}[language=bash]
make check
\end{lstlisting}

in the main FLINT directory after configuring FLINT.

\chapter{Reporting bugs}

The maintainer wishes to be made aware of any and all bugs.  Please send an 
email with your bug report to \url{hart_wb@yahoo.com} or report them on the
FLINT devel list \url{https://groups.google.com/group/flint-devel?hl=en}.

If possible please include details of your system, the version of GCC, 
the versions of GMP/MPIR and MPFR as well as precise details of how to 
replicate the bug.

Note that FLINT needs to be linked against version 2.6.0 or later of MPIR
(or version 5.1.1 or later of GMP), version 3.0.0 or later of MPFR and
must be compiled with gcc version 2.96 or later.  

\chapter{Contributors}

FLINT has been developed since 2007 by a large number of people. Initially
the library was started by David Harvey and William Hart. Later maintenance
of the library was taken over solely by William Hart.

The main authors of FLINT to date have been William Hart, David Harvey (no
longer active), Fredrik Johansson, Sebastian Pancratz and Andy Novocin.

Other significant contributions to FLINT have been made by Jason Papadopoulos,
Gonzalo Tornaria, David Howden, Burcin Erocal, Tom Boothby, Daniel Woodhouse, 
Tomasz Lechowski, Richard Howell-Peak, Peter Shrimpton, Andr�s Goens, Lina
Kulakova, Thomas DuBuisson, Jean-Pierre Flori, Frithjof Schulze, Curtis 
Bright.

Jan Tuitman contributed to the design of the padics module.

Additional research was contributed by Daniel Scott and Daniel Ellam.

Further patches and bug reports have been made by Michael Abshoff, Didier 
Deshommes, Craig Citro, Timothy Abbot, Carl Witty, Jaap Spies, Kiran 
Kedlaya, William Stein, Kate Minola, Robert Bradshaw, Serge Torres, Dan 
Grayson, Martin Lee, Bob Smith, Antony Vennard, Fr�d�ric Chyzak, Julien
Puydt and many others.

Ralf Hemmecke made available an autotools build for FLINT.

Numerous people have contributed to wrapping FLINT in Sage and debugging, 
including Mike Hansen, Jean-Pierre Flori, Burcin Erocal, Robert Bradshaw,
Martin Albrecht, Sebastian Pancratz, Fredrik Johansson,

Some code (\code{longlong.h} and \code{clz_tab.c}) has been used from
an LGPL v2+ version of the GMP library. The main author of the GMP 
library is Torbjorn Granlund.

FLINT 2 was a complete rewrite from scratch which began in about 2010.
 
\chapter{Tuning FLINT}

FLINT uses a highly optimised Fast Fourier Transform routine for 
polynomial multiplication and some integer multiplication routines. This
can be tuned by first typing \code{make tune} and then running the 
program \code{build/fft/tune/tune_fft}. 

The output of the program can be pasted into  \code{fft_tuning64.in} or 
\code{fft_tuning32.in} depending on the ABI of the current platform. FLINT 
must then be configured again and a clean build initiated. 

Tuning is only necessary if you suspect that very large polynomial and 
integer operations (millions of bits) are taking longer than they should.

\chapter{Example programs}

FLINT comes with example programs to demonstrate current and future FLINT 
features.  To build the example programs, type:

\begin{lstlisting}[language=bash]
make examples
\end{lstlisting}

The example programs are built in the \code{build/examples} directory.
You must set your \code{LD_LIBRARY_PATH} or equivalent for the flint, mpir
and mpfr libraries. See your operating system documentation to see how to
set this.

The current example programs are:

\code{partitions} Demonstrates the partition counting code, e.g.\\
\code{build/examples/partitions 1000000000} will compute the number of 
partitions of \code{10^9}.

\code{delta_qexp}  Computes the $n$-th term of the delta function, e.g.\\ 
\code{build/examples/delta_qexp 1000000} will compute the one million-th 
term of the $q$-expansion of delta.

\code{crt}  Demonstrates the integer Chinese Remainder code, e.g.\ 
\code{build/examples/crt 10382788} will build up the given integer from 
its value mod various primes.

\code{multi_crt}  Demonstrates the fast tree version of the integer Chinese 
Remainder code, e.g.\ 
\code{build/examples/multi_crt 100493287498239 13} will build up the given 
integer from its value mod the given number of primes.

\code{stirling_matrix}  Generates Stirling number matrices of the first
and second kind and computes their product, which should come out as the
identity matrix. The matrices are printed to standard output.
For example \code{build/examples/stirling_matrix 10} does this
with 10 by 10 matrices.

\code{fmpz_poly_factor_zassenhaus} Demonstrates the factorisation of a
small polynomial. A larger polynomials is also provided on disk and a
small (obvious) change to the example program will read this file instead
of using the hard coded polynomial.

\code{padic} Gives examples of the usage of many functions in the padic
module.

\code{fmpz_poly_q} Gives a very simple example of the \code{fmpz_poly_q}
module.

\code{fmpz_poly} Gives a very simple example of the \code{fmpz_poly}
module.

\code{fmpq_poly} Gives a very simple example of the \code{fmpq_poly}
module.

\chapter{FLINT macros}

The file \code{flint.h} contains various useful macros.

The macro constant \code{FLINT_BITS} is set at compile time to be the 
number of bits per limb on the machine.  FLINT requires it to be either 
32 or 64 bits.  Other architectures are not currently supported.

The macro constant \code{FLINT_D_BITS} is set at compile time to be the 
number of bits per double on the machine or one less than the number of 
bits per limb, whichever is smaller.  This will have the value 53 or 31 
on currently supported architectures.  Numerous internal functions using 
precomputed inverses only support operands up to \code{FLINT_D_BITS} bits, 
hence the macro.

The macro \code{FLINT_ABS(x)} returns the absolute value of \code{x}
for primitive signed numerical types.  It might fail for least negative 
values such as \code{INT_MIN} and \code{LONG_MIN}.

The macro \code{FLINT_MIN(x, y)} returns the minimum of \code{x} and 
\code{y} for primitive signed or unsigned numerical types.  This macro 
is only safe to use when \code{x} and \code{y} are of the same type, 
to avoid problems with integer promotion.

Similar to the previous macro, \code{FLINT_MAX(x, y)} returns the 
maximum of \code{x} and \code{y}.

The function \code{FLINT_BIT_COUNT(x)} returns the number of binary bits 
required to represent an \code{ulong x}.  If \code{x} is zero, 
returns~$0$.

Derived from this there are the two macros \code{FLINT_FLOG2(x)} and 
\code{FLINT_CLOG2(x)} which, for any $x \geq 1$, compute $\floor{\log_2{x}}$ 
and $\ceil{\log_2{x}}$.

%%%%%%%%%%%%%%%%%%%%%%%%%%%%%%%%%%%%%%%%%%%%%%%%%%%%%%%%%%%%%%%%%%%%%%%%%%%%%%%%
% Integers                                                                     %
%%%%%%%%%%%%%%%%%%%%%%%%%%%%%%%%%%%%%%%%%%%%%%%%%%%%%%%%%%%%%%%%%%%%%%%%%%%%%%%%

\chapter{fmpz}
\epigraph{Arbitrary precision integers}{}

\section{Introduction}

By default, an \code{fmpz_t} is implemented as an array of \code{fmpz}'s of 
length one to allow passing by reference as one can do with GMP/ MPIR's 
\code{mpz_t} type.  The \code{fmpz_t} type is simply a single limb, though 
the user does not need to be aware of this except in one specific case 
outlined below.

In all respects, \code{fmpz_t}'s act precisely like GMP/ MPIR's 
\code{mpz_t}'s, with automatic memory management, however, in the first 
place only one limb is used to implement them.  Once an \code{fmpz_t} 
overflows a limb then a multiprecision integer is automatically allocated 
and instead of storing the actual integer data the \code{slong} which 
implements the type becomes an index into a FLINT wide array of 
\code{mpz_t}'s.

These internal implementation details are not important for the user to 
understand, except for three important things.

Firstly, \code{fmpz_t}'s will be more efficient than \code{mpz_t}'s for 
single limb operations, or more precisely for signed quantities whose 
absolute value does not exceed \code{FLINT_BITS - 2} bits.

Secondly, for small integers that fit into \code{FLINT_BITS - 2} bits 
much less memory will be used than for an \code{mpz_t}.  When very many 
\code{fmpz_t}'s are used, there can be important cache benefits on 
account of this.

Thirdly, it is important to understand how to deal with arrays of 
\code{fmpz_t}'s.  As for \code{mpz_t}'s, there is an underlying type, 
an \code{fmpz}, which can be used to create the array, e.g.\ 
\begin{lstlisting}
fmpz myarr[100];
\end{lstlisting}
Now recall that an \code{fmpz_t} is an array of length one of \code{fmpz}'s.
Thus, a pointer to an \code{fmpz} can be used in place of an \code{fmpz_t}.
For example, to find the sign of the third integer in our array we would 
write 
\begin{lstlisting}
int sign = fmpz_sgn(myarr + 2);
\end{lstlisting}

The \code{fmpz} module provides routines for memory management, basic 
manipulation and basic arithmetic.

Unless otherwise specified, all functions in this section permit aliasing 
between their input arguments and between their input and output 
arguments.

\section{Simple example}

The following example computes the square of the integer $7$ and prints 
the result.
\begin{lstlisting}[language=c]
#include "fmpz.h"
...
fmpz_t x, y;
fmpz_init(x);
fmpz_init(y);
fmpz_set_ui(x, 7);
fmpz_mul(y, x, x);
fmpz_print(x);
printf("^2 = ");
fmpz_print(y);
printf("\n");
fmpz_clear(x);
fmpz_clear(y);
\end{lstlisting}

The output is:
\begin{lstlisting}
7^2 = 49
\end{lstlisting}

We now describe the functions available in the \code{fmpz} module.

\input{input/fmpz.tex}

%%%%%%%%%%%%%%%%%%%%%%%%%%%%%%%%%%%%%%%%%%%%%%%%%%%%%%%%%%%%%%%%%%%%%%%%%%%%%%%%
% Integer vectors                                                              %
%%%%%%%%%%%%%%%%%%%%%%%%%%%%%%%%%%%%%%%%%%%%%%%%%%%%%%%%%%%%%%%%%%%%%%%%%%%%%%%%

\chapter{fmpz\_vec}
\epigraph{Vectors over $\Z$}{}

\input{input/fmpz_vec.tex}

%%%%%%%%%%%%%%%%%%%%%%%%%%%%%%%%%%%%%%%%%%%%%%%%%%%%%%%%%%%%%%%%%%%%%%%%%%%%%%%%
% Integer factorisation                                                        %
%%%%%%%%%%%%%%%%%%%%%%%%%%%%%%%%%%%%%%%%%%%%%%%%%%%%%%%%%%%%%%%%%%%%%%%%%%%%%%%%

\chapter{fmpz\_factor}
\epigraph{Factorisation in $\Z$}{}

\input{input/fmpz_factor.tex}

%%%%%%%%%%%%%%%%%%%%%%%%%%%%%%%%%%%%%%%%%%%%%%%%%%%%%%%%%%%%%%%%%%%%%%%%%%%%%%%%
% Integer matrices                                                             %
%%%%%%%%%%%%%%%%%%%%%%%%%%%%%%%%%%%%%%%%%%%%%%%%%%%%%%%%%%%%%%%%%%%%%%%%%%%%%%%%

\chapter{fmpz\_mat}
\epigraph{Matrices over $\Z$}{}

\section{Introduction}

The \code{fmpz_mat_t} data type represents dense matrices of multiprecision
integers, implemented using \code{fmpz} vectors.

No automatic resizing is performed: in general, the user must provide
matrices of correct dimensions for both input and output variables. Output
variables are \emph{not} allowed to be aliased with input variables unless
otherwise noted.

Matrices are indexed from zero: an $m \times n$ matrix
has rows of index $0,1,\ldots,m-1$ and columns of
index $0,1,\ldots,n-1$. One or both of $m$ and $n$ may be zero.

Elements of a matrix can be read or written using the \code{fmpz_mat_entry}
macro, which returns a reference to the entry at a given row and column index.
This reference can be passed as an input or output \code{fmpz_t} variable to 
any function in the \code{fmpz} module for direct manipulation.

\section{Simple example}
The following example creates the $2 \times 2$ matrix $A$ with
value $2i+j$ at row~$i$ and column~$j$, computes $B = A^2$,
and prints both matrices.

\begin{lstlisting}[language=c]
#include "fmpz.h"
#include "fmpz_mat.h"
...
long i, j;
fmpz_mat_t A;
fmpz_mat_t B;
fmpz_mat_init(A, 2, 2);
fmpz_mat_init(B, 2, 2);
for (i = 0; i < 2; i++)
    for (j = 0; j < 2; j++)
        fmpz_set_ui(fmpz_mat_entry(A, i, j), 2*i+j);
fmpz_mat_mul(B, A, A);
printf("A = \n");
fmpz_mat_print_pretty(A);
printf("A^2 = \n");
fmpz_mat_print_pretty(B);
fmpz_mat_clear(A);
fmpz_mat_clear(B);
\end{lstlisting}

The output is:
\begin{lstlisting}
A = 
[[0 1]
[2 3]]
A^2 = 
[[2 3]
[6 11]]
\end{lstlisting}


\input{input/fmpz_mat.tex}

%%%%%%%%%%%%%%%%%%%%%%%%%%%%%%%%%%%%%%%%%%%%%%%%%%%%%%%%%%%%%%%%%%%%%%%%%%%%%%%%
% Integer polynomials                                                          %
%%%%%%%%%%%%%%%%%%%%%%%%%%%%%%%%%%%%%%%%%%%%%%%%%%%%%%%%%%%%%%%%%%%%%%%%%%%%%%%%

\chapter{fmpz\_poly}
\epigraph{Polynomials over $\Z$}{}

\section{Introduction}

The \code{fmpz_poly_t} data type represents elements of $\Z[x]$. The 
\code{fmpz_poly} module provides routines for memory management, basic 
arithmetic, and conversions from or to other types.

Each coefficient of an \code{fmpz_poly_t} is an integer of the FLINT 
\code{fmpz_t} type.  There are two advantages of this model.  Firstly, 
the \code{fmpz_t} type is memory managed, so the user can manipulate 
individual coefficients of a polynomial without having to deal with 
tedious memory management.  Secondly, a coefficient of an 
\code{fmpz_poly_t} can be changed without changing the size of any 
of the other coefficients.

Unless otherwise specified, all functions in this section permit aliasing 
between their input arguments and between their input and output arguments.

\section{Simple example}

The following example computes the square of the polynomial $5x^3 - 1$.
\begin{lstlisting}[language=c]
#include "fmpz_poly.h"
...
fmpz_poly_t x, y;
fmpz_poly_init(x);
fmpz_poly_init(y);
fmpz_poly_set_coeff_ui(x, 3, 5);
fmpz_poly_set_coeff_si(x, 0, -1);
fmpz_poly_mul(y, x, x);
fmpz_poly_print(x); printf("\n");
fmpz_poly_print(y); printf("\n");
fmpz_poly_clear(x);
fmpz_poly_clear(y);
\end{lstlisting}

The output is:
\begin{lstlisting}
4  -1 0 0 5
7  1 0 0 -10 0 0 25
\end{lstlisting}

\section{Definition of the fmpz\_poly\_t type}

The \code{fmpz_poly_t} type is a typedef for an array of length~1 of 
\code{fmpz_poly_struct}'s.  This permits passing parameters of type 
\code{fmpz_poly_t} by reference in a manner similar to the way GMP integers 
of type \code{mpz_t} can be passed by reference. 

In reality one never deals directly with the \code{struct} and simply deals 
with objects of type \code{fmpz_poly_t}.  For simplicity we will think of an 
\code{fmpz_poly_t} as a \code{struct}, though in practice to access fields 
of this \code{struct}, one needs to dereference first, e.g.\ to access the 
\code{length} field of an \code{fmpz_poly_t} called \code{poly1} one writes 
\code{poly1->length}. 

An \code{fmpz_poly_t} is said to be \emph{normalised} if either 
\code{length} is zero, or if the leading coefficient of the polynomial is 
non-zero.  All \code{fmpz_poly} functions expect their inputs to be 
normalised, and unless otherwise specified they produce output that is 
normalised. 

It is recommended that users do not access the fields of an 
\code{fmpz_poly_t} or its coefficient data directly, but make use of the 
functions designed for this purpose, detailed below.

Functions in \code{fmpz_poly} do all the memory management for the user. 
One does not need to specify the maximum length or number of limbs per 
coefficient in advance before using a polynomial object.  FLINT reallocates 
space automatically as the computation proceeds, if more space is required. 
Each coefficient is also managed separately, being resized as needed, 
independently of the other coefficients.

We now describe the functions available in \code{fmpz_poly}.

\input{input/fmpz_poly.tex}

%%%%%%%%%%%%%%%%%%%%%%%%%%%%%%%%%%%%%%%%%%%%%%%%%%%%%%%%%%%%%%%%%%%%%%%%%%%%%%%%
% Integer polynomial factorisation                                             %
%%%%%%%%%%%%%%%%%%%%%%%%%%%%%%%%%%%%%%%%%%%%%%%%%%%%%%%%%%%%%%%%%%%%%%%%%%%%%%%%

\chapter{fmpz\_poly\_factor}
\epigraph{Factorisation of polynomials over $\Z$}{}

\input{input/fmpz_poly_factor.tex}

%%%%%%%%%%%%%%%%%%%%%%%%%%%%%%%%%%%%%%%%%%%%%%%%%%%%%%%%%%%%%%%%%%%%%%%%%%%%%%%%
% Rational numbers                                                             %
%%%%%%%%%%%%%%%%%%%%%%%%%%%%%%%%%%%%%%%%%%%%%%%%%%%%%%%%%%%%%%%%%%%%%%%%%%%%%%%%

\chapter{fmpq}
\epigraph{Arbitrary-precision rational numbers}{}

\section{Introduction}

The \code{fmpq_t} data type represents rational numbers
as fractions of multiprecision integers.

An \code{fmpq_t} is an array of length 1 of type \code{fmpq},
with \code{fmpq} being implemented as a pair of \code{fmpz}'s
representing numerator and denominator.

This format is designed to allow rational numbers with small
numerators or denominators to be stored and manipulated
efficiently. When components no longer fit in single
machine words, the cost of \code{fmpq_t} arithmetic
is roughly the same as that of \code{mpq_t} arithmetic,
plus a small amount of overhead.

A fraction is said to be in canonical form if the numerator
and denominator have no common factor and the denominator is
positive. Except where otherwise noted, all functions in the
\code{fmpq} module assume that
inputs are in canonical form, and produce outputs in canonical form.
The user can manipulate the numerator and denominator of an
\code{fmpq_t} as arbitrary integers, but then becomes
responsible for canonicalising the number (for example by calling
\code{fmpq_canonicalise}) before passing it to any library function.

For most operations, both a function operating
on \code{fmpq_t}'s and an underscore version operating
on \code{fmpz_t} components are provided. The underscore
functions may perform less error checking,
and may impose limitations on aliasing between the
input and output variables, but 
generally assume that the components are in
canonical form just like the non-underscore functions.

\input{input/fmpq.tex}

%%%%%%%%%%%%%%%%%%%%%%%%%%%%%%%%%%%%%%%%%%%%%%%%%%%%%%%%%%%%%%%%%%%%%%%%%%%%%%%%
% Rational matrices                                                            %
%%%%%%%%%%%%%%%%%%%%%%%%%%%%%%%%%%%%%%%%%%%%%%%%%%%%%%%%%%%%%%%%%%%%%%%%%%%%%%%%

\chapter{fmpq\_mat}
\epigraph{Matrices over $\Q$}{}

\section{Introduction}

The \code{fmpq_mat_t} data type represents matrices over $\Q$.

A rational matrix is stored as an array of \code{fmpq} elements in order
to allow convenient and efficient manipulation of individual entries.
In general, \code{fmpq_mat} functions assume that input entries are
in canonical form, and produce output with entries in canonical form.

Since rational arithmetic is expensive, computations are typically performed
by clearing denominators, performing the heavy work over the integers,
and converting the final result back to a rational matrix. The
\code{fmpq_mat} functions take care of such conversions transparently.
For users who need fine-grained control, various
functions for conversion between rational and integer matrices are provided.

\input{input/fmpq_mat.tex}

%%%%%%%%%%%%%%%%%%%%%%%%%%%%%%%%%%%%%%%%%%%%%%%%%%%%%%%%%%%%%%%%%%%%%%%%%%%%%%%%
% Rational polynomials                                                         %
%%%%%%%%%%%%%%%%%%%%%%%%%%%%%%%%%%%%%%%%%%%%%%%%%%%%%%%%%%%%%%%%%%%%%%%%%%%%%%%%

\chapter{fmpq\_poly}
\epigraph{Polynomials over $\Q$}{}

\section{Introduction}

The \code{fmpq_poly_t} data type represents elements of $\Q[x]$. The 
\code{fmpq_poly} module provides routines for memory management, basic 
arithmetic, and conversions from or to other types.

A rational polynomial is stored as the quotient of an integer polynomial 
and an integer denominator.  To be more precise, the coefficient vector 
of the numerator can be accessed with the function \code{fmpq_poly_numref()} 
and the denominator with \code{fmpq_poly_denref()}.  Although one can 
construct use cases in which a representation as a list of rational 
coefficients would be beneficial, the choice made here is typically 
more efficient.

We can obtain a unique representation based on this choice by enforcing, 
for non-zero polynomials, that the numerator and denominator are coprime 
and that the denominator is positive.  The unique representation of the 
zero polynomial is chosen as $0/1$.

Similar to the situation in the \code{fmpz_poly_t} case, an 
\code{fmpq_poly_t} object also has a \code{length} parameter, which 
denotes the length of the vector of coefficients of the numerator. 
We say a polynomial is \emph{normalised} either if this length is zero 
or if the leading coefficient is non-zero.

We say a polynomial is in \emph{canonical} form if it is given in the 
unique representation discussed above and normalised.

The functions provided in this module roughly fall into two categories:

On the one hand, there are functions mainly provided for the user, whose 
names do not begin with an underscore.  These typically operate on 
polynomials of type \code{fmpq_poly_t} in canonical form and, unless 
specified otherwise, permit aliasing between their input arguments and 
between their output arguments.

On the other hand, there are versions of these functions whose names are 
prefixed with a single underscore.  These typically operate on polynomials 
given in the form of a triple of object of types \code{fmpz *}, 
\code{fmpz_t}, and \code{slong}, containing the numerator, denominator and 
length, respectively.  In general, these functions expect their input to 
be normalised, i.e.\ they do not allow zero padding, and to be in lowest 
terms, and they do not allow their input and output arguments to be aliased.

\input{input/fmpq_poly.tex}

%%%%%%%%%%%%%%%%%%%%%%%%%%%%%%%%%%%%%%%%%%%%%%%%%%%%%%%%%%%%%%%%%%%%%%%%%%%%%%%%
% Rational functions                                                           %
%%%%%%%%%%%%%%%%%%%%%%%%%%%%%%%%%%%%%%%%%%%%%%%%%%%%%%%%%%%%%%%%%%%%%%%%%%%%%%%%

\chapter{fmpz\_poly\_q}
\epigraph{Rational functions over $\Q$}{}

\section{Introduction}

The module \code{fmpz_poly_q} provides functions for performing 
arithmetic on rational functions in $\mathbf{Q}(t)$, represented as 
quotients of integer polynomials of type \code{fmpz_poly_t}.  These 
functions start with the prefix \code{fmpz_poly_q_}.

Rational functions are stored in objects of type \code{fmpz_poly_q_t}, 
which is an array of \code{fmpz_poly_q_struct}'s of length one.  This 
permits passing parameters of type \code{fmpz_poly_q_t} by reference.  

The representation of a rational function as the quotient of two integer 
polynomials can be made canonical by demanding the numerator and 
denominator to be coprime (as integer polynomials) and the denominator to 
have positive leading coefficient.  As the only special case, we represent 
the zero function as $0/1$.  All arithmetic functions assume that the 
operands are in this canonical form, and canonicalize their result.  If the 
numerator or denominator is modified individually, for example using the 
macros \code{fmpz_poly_q_numref()} and \code{fmpz_poly_q_denref()}, 
it is the user's responsibility to canonicalise the rational function 
using the function \code{fmpz_poly_q_canonicalise()} if necessary.

All methods support aliasing of their inputs and outputs \emph{unless} 
explicitly stated otherwise, subject to the following caveat.  If 
different rational functions (as objects in memory, not necessarily in the 
mathematical sense) share some of the underlying integer polynomial 
objects, the behaviour is undefined.

The basic arithmetic operations, addition, subtraction and multiplication, 
are all implemented using adapted versions of Henrici's algorithms, 
see~\citep{Hen1956}.  Differentiation is implemented in a way slightly 
improving on the algorithm described in~\citep{Hor1972}.

\section{Simple example}


The following example computes the product of two rational functions and 
prints the result:
\begin{lstlisting}[language=c]
#include "fmpz_poly_q.h"
...
char *str, *strf, *strg;
fmpz_poly_q_t f, g;
fmpz_poly_q_init(f);
fmpz_poly_q_init(g);
fmpz_poly_q_set_str(f, "2  1 3/1  2");
fmpz_poly_q_set_str(g, "1  3/2  2 7");
strf = fmpz_poly_q_get_str_pretty(f, "t");
strg = fmpz_poly_q_get_str_pretty(g, "t");
fmpz_poly_q_mul(f, f, g);
str  = fmpz_poly_q_get_str_pretty(f, "t");
printf("%s * %s = %s\n", strf, strg, str);
free(str);
free(strf);
free(strg);
fmpz_poly_q_clear(f);
fmpz_poly_q_clear(g);
\end{lstlisting}

The output is:
\begin{lstlisting}[language=c]
(3*t+1)/2 * 3/(7*t+2) = (9*t+3)/(14*t+4)
\end{lstlisting}

\input{input/fmpz_poly_q.tex}

%%%%%%%%%%%%%%%%%%%%%%%%%%%%%%%%%%%%%%%%%%%%%%%%%%%%%%%%%%%%%%%%%%%%%%%%%%%%%%%%
% Matrices over integer polynomials                                            %
%%%%%%%%%%%%%%%%%%%%%%%%%%%%%%%%%%%%%%%%%%%%%%%%%%%%%%%%%%%%%%%%%%%%%%%%%%%%%%%%

\chapter{fmpz\_poly\_mat}
\epigraph{Matrices over $\mathbf{Z}[x]$}{}

The \code{fmpz_poly_mat_t} data type represents matrices whose
entries are integer polynomials.

The \code{fmpz_poly_mat_t} type is defined as an array of
\code{fmpz_poly_mat_struct}'s of length one.
This  permits passing parameters of type \code{fmpz_poly_mat_t}
by reference.  

An integer polynomial matrix internally consists of a single array
of \code{fmpz_poly_struct}'s, representing a dense matrix in
row-major order. This array is only directly indexed
during memory allocation and deallocation. A separate array
holds pointers to the start of each row, and is used for all
indexing. This allows the rows of a matrix to be permuted
quickly by swapping pointers.

Matrices having zero rows or columns are allowed.

The shape of a matrix is fixed upon initialisation.
The user is assumed to provide input and output variables
whose dimensions are compatible with the given operation.

\section{Simple example}

The following example constructs the matrix
$\begin{pmatrix} 2x+1 & x \\ 1-x & -1 \end{pmatrix}$ and computes
its determinant.

\begin{lstlisting}[language=c]
#include "fmpz_poly.h"
#include "fmpz_poly_mat.h"
...
fmpz_poly_mat_t A;
fmpz_poly_t P;

fmpz_poly_mat_init(A, 2, 2);
fmpz_poly_init(P);

fmpz_poly_set_str(fmpz_poly_mat_entry(A, 0, 0), "2  1 2");
fmpz_poly_set_str(fmpz_poly_mat_entry(A, 0, 1), "2  0 1");
fmpz_poly_set_str(fmpz_poly_mat_entry(A, 1, 0), "2  1 -1");
fmpz_poly_set_str(fmpz_poly_mat_entry(A, 1, 1), "1  -1");

fmpz_poly_mat_det(P, A);
fmpz_poly_print_pretty(P, "x");

fmpz_poly_clear(P);
fmpz_poly_mat_clear(A);
\end{lstlisting}

The output is:
\begin{lstlisting}[language=c]
x^2-3*x-1
\end{lstlisting}

\input{input/fmpz_poly_mat.tex}

%%%%%%%%%%%%%%%%%%%%%%%%%%%%%%%%%%%%%%%%%%%%%%%%%%%%%%%%%%%%%%%%%%%%%%%%%%%%%%%%
% Vectors over Z / nZ for word-sized moduli                                    %
%%%%%%%%%%%%%%%%%%%%%%%%%%%%%%%%%%%%%%%%%%%%%%%%%%%%%%%%%%%%%%%%%%%%%%%%%%%%%%%%

\chapter{nmod\_vec}
\epigraph{Vectors over $\Z / n \Z$ for word-sized moduli}{}

\input{input/nmod_vec.tex}

%%%%%%%%%%%%%%%%%%%%%%%%%%%%%%%%%%%%%%%%%%%%%%%%%%%%%%%%%%%%%%%%%%%%%%%%%%%%%%%%
% Polynomials over Z / nZ for word-sized moduli                                %
%%%%%%%%%%%%%%%%%%%%%%%%%%%%%%%%%%%%%%%%%%%%%%%%%%%%%%%%%%%%%%%%%%%%%%%%%%%%%%%%

\chapter{nmod\_poly}
\epigraph{Polynomials over $\Z / n \Z$ for word-sized moduli}{}

\section{Introduction}

The \code{nmod_poly_t} data type represents elements of $\Z/n\Z[x]$ for
a fixed modulus $n$. The \code{nmod_poly} module provides routines for 
memory management, basic arithmetic and some higher level functions
such as GCD, etc.

Each coefficient of an \code{nmod_poly_t} is of type \code{mp_limb_t}
and represents an integer reduced modulo the fixed modulus $n$.  

Unless otherwise specified, all functions in this section permit aliasing 
between their input arguments and between their input and output arguments.

\section{Simple example}

The following example computes the square of the polynomial $5x^3 + 6$
in $\Z/7\Z[x]$.
\begin{lstlisting}[language=c]
#include "nmod_poly.h"
...
nmod_poly_t x, y;
nmod_poly_init(x, 7);
nmod_poly_init(y, 7);
nmod_poly_set_coeff_ui(x, 3, 5);
nmod_poly_set_coeff_ui(x, 0, 6);
nmod_poly_mul(y, x, x);
nmod_poly_print(x); printf("\n");
nmod_poly_print(y); printf("\n");
nmod_poly_clear(x);
nmod_poly_clear(y);
\end{lstlisting}

The output is:
\begin{lstlisting}
4 7  6 0 0 5
7 7  1 0 0 4 0 0 4
\end{lstlisting}

\section{Definition of the nmod\_poly\_t type}

The \code{nmod_poly_t} type is a typedef for an array of length~1 of 
\code{nmod_poly_struct}'s.  This permits passing parameters of type 
\code{nmod_poly_t} by reference. 

In reality one never deals directly with the \code{struct} and simply deals 
with objects of type \code{nmod_poly_t}.  For simplicity we will think of an 
\code{nmod_poly_t} as a \code{struct}, though in practice to access fields 
of this \code{struct}, one needs to dereference first, e.g.\ to access the 
\code{length} field of an \code{nmod_poly_t} called \code{poly1} one writes 
\code{poly1->length}. 

An \code{nmod_poly_t} is said to be \emph{normalised} if either 
\code{length} is zero, or if the leading coefficient of the polynomial is 
non-zero.  All \code{nmod_poly} functions expect their inputs to be 
normalised and for all coefficients to be reduced modulo $n$, and unless 
otherwise specified they produce output that is normalised with 
coefficients reduced modulo $n$.

It is recommended that users do not access the fields of an 
\code{nmod_poly_t} or its coefficient data directly, but make use of the 
functions designed for this purpose, detailed below.

Functions in \code{nmod_poly} do all the memory management for the user. 
One does not need to specify the maximum length in advance before using a 
polynomial object.  FLINT reallocates space automatically as the computation 
proceeds, if more space is required. 

We now describe the functions available in \code{nmod_poly}.

\input{input/nmod_poly.tex}

%%%%%%%%%%%%%%%%%%%%%%%%%%%%%%%%%%%%%%%%%%%%%%%%%%%%%%%%%%%%%%%%%%%%%%%%%%%%%%%%
% Matrices over Z / nZ for word-sized moduli                                   %
%%%%%%%%%%%%%%%%%%%%%%%%%%%%%%%%%%%%%%%%%%%%%%%%%%%%%%%%%%%%%%%%%%%%%%%%%%%%%%%%

\chapter{nmod\_mat}
\epigraph{Matrices over $\Z / n \Z$ for word-sized moduli}{}

\section{Introduction}

An \code{nmod_mat_t} represents a matrix of integers modulo $n$, for any 
non-zero modulus $n$ that fits in a single limb,
up to $2^{32}-1$ or $2^{64}-1$.

The \code{nmod_mat_t} type is defined as an array of
\code{nmod_mat_struct}'s of length one.
This  permits passing parameters of type \code{nmod_mat_t}
by reference.

An \code{nmod_mat_t} internally consists of a single array of
\code{mp_limb_t}'s, representing a dense matrix in row-major order.
This array is only directly indexed during memory allocation and
deallocation. A separate array holds pointers to the start of each
row, and is used for all indexing. This allows the rows of a matrix
to be permuted quickly by swapping pointers.

Matrices having zero rows or columns are allowed.

The shape of a matrix is fixed upon initialisation.
The user is assumed to provide input and output variables
whose dimensions are compatible with the given operation.

It is assumed that all matrices passed to a function have the same modulus.
The modulus is assumed to be a prime number in functions that
perform some kind of division, solving, or Gaussian elimination
(including computation of rank and determinant),
but can be composite in functions that only perform basic manipulation
and ring operations (e.g. transpose and matrix multiplication).

The user can manipulate matrix entries directly, but must
assume responsibility for normalising all values to the range $[0, n)$.

\input{input/nmod_mat.tex}

%%%%%%%%%%%%%%%%%%%%%%%%%%%%%%%%%%%%%%%%%%%%%%%%%%%%%%%%%%%%%%%%%%%%%%%%%%%%%%%%
% Matrices over integer polynomials mod n                                      %
%%%%%%%%%%%%%%%%%%%%%%%%%%%%%%%%%%%%%%%%%%%%%%%%%%%%%%%%%%%%%%%%%%%%%%%%%%%%%%%%

\chapter{nmod\_poly\_mat}
\epigraph{Matrices over $\Z / n \Z[x]$ for word-sized moduli}{}

The \code{nmod_poly_mat_t} data type represents matrices whose
entries are polynomials having coefficients in $\Z / n \Z$.
We generally assume that $n$ is a prime number.

The \code{nmod_poly_mat_t} type is defined as an array of
\code{nmod_poly_mat_struct}'s of length one.
This permits passing parameters of type \code{nmod_poly_mat_t}
by reference.

A matrix internally consists of a single array
of \code{nmod_poly_struct}'s, representing a dense matrix in
row-major order. This array is only directly indexed
during memory allocation and deallocation. A separate array
holds pointers to the start of each row, and is used for all
indexing. This allows the rows of a matrix to be permuted
quickly by swapping pointers.

Matrices having zero rows or columns are allowed.

The shape of a matrix is fixed upon initialisation.
The user is assumed to provide input and output variables
whose dimensions are compatible with the given operation.

\input{input/nmod_poly_mat.tex}

%%%%%%%%%%%%%%%%%%%%%%%%%%%%%%%%%%%%%%%%%%%%%%%%%%%%%%%%%%%%%%%%%%%%%%%%%%%%%%%%
% Polynomials over Z/nZ for general moduli                                     %
%%%%%%%%%%%%%%%%%%%%%%%%%%%%%%%%%%%%%%%%%%%%%%%%%%%%%%%%%%%%%%%%%%%%%%%%%%%%%%%%

\chapter{fmpz\_mod\_poly}
\epigraph{Polynomials over $\Z / n \Z$ for general moduli}{}

\section{Introduction}

The \code{fmpz_mod_poly_t} data type represents elements of $\Z/n\Z[x]$ for
a fixed modulus $n$. The \code{fmpz_mod_poly} module provides routines for 
memory management, basic arithmetic and some higher level functions
such as GCD, etc.

Each coefficient of an \code{fmpz_mod_poly_t} is of type \code{fmpz} 
and represents an integer reduced modulo the fixed modulus $n$ in the 
range $[0,n)$.

Unless otherwise specified, all functions in this section permit aliasing 
between their input arguments and between their input and output arguments.

\section{Simple example}

The following example computes the square of the polynomial $5x^3 + 6$
in $\Z/7\Z[x]$.
\begin{lstlisting}[language=c]
#include "fmpz_mod_poly.h"
...
fmpz_t n;
fmpz_mod_poly_t x, y;

fmpz_init_set_ui(n, 7);
fmpz_mod_poly_init(x, n);
fmpz_mod_poly_init(y, n);
fmpz_mod_poly_set_coeff_ui(x, 3, 5);
fmpz_mod_poly_set_coeff_ui(x, 0, 6);
fmpz_mod_poly_sqr(y, x);
fmpz_mod_poly_print(x); printf("\n");
fmpz_mod_poly_print(y); printf("\n");
fmpz_mod_poly_clear(x);
fmpz_mod_poly_clear(y);
fmpz_clear(n);
\end{lstlisting}

The output is:
\begin{lstlisting}
4 7  6 0 0 5
7 7  1 0 0 4 0 0 4
\end{lstlisting}

\section{Definition of the fmpz\_mod\_poly\_t type}

The \code{fmpz_mod_poly_t} type is a typedef for an array of length~1 of\\ 
\code{fmpz_mod_poly_struct}'s.  This permits passing parameters of type 
\code{fmpz_mod_poly_t} by reference. 

In reality one never deals directly with the \code{struct} and simply deals 
with objects of type \code{fmpz_mod_poly_t}.  For simplicity we will think of an 
\code{fmpz_mod_poly_t} as a \code{struct}, though in practice to access fields 
of this \code{struct}, one needs to dereference first, e.g.\ to access the 
\code{length} field of an \code{fmpz_mod_poly_t} called \code{poly1} one writes 
\code{poly1->length}. 

An \code{fmpz_mod_poly_t} is said to be \emph{normalised} if either 
\code{length} is zero, or if the leading coefficient of the polynomial is 
non-zero.  All \code{fmpz_mod_poly} functions expect their inputs to be 
normalised and all coefficients to be reduced modulo~$n$, and unless 
otherwise specified they produce output that is normalised with coefficients 
reduced modulo~$n$.

It is recommended that users do not access the fields of an 
\code{fmpz_mod_poly_t} or its coefficient data directly, but make use of the 
functions designed for this purpose, detailed below.

Functions in \code{fmpz_mod_poly} do all the memory management for the user. 
One does not need to specify the maximum length in advance before using a 
polynomial object.  FLINT reallocates space automatically as the computation 
proceeds, if more space is required. 

We now describe the functions available in \code{fmpz_mod_poly}.

\input{input/fmpz_mod_poly.tex}

%%%%%%%%%%%%%%%%%%%%%%%%%%%%%%%%%%%%%%%%%%%%%%%%%%%%%%%%%%%%%%%%%%%%%%%%%%%%%%%%
% Padic numbers                                                                %
%%%%%%%%%%%%%%%%%%%%%%%%%%%%%%%%%%%%%%%%%%%%%%%%%%%%%%%%%%%%%%%%%%%%%%%%%%%%%%%%

\chapter{padic}
\epigraph{$p$-Adic numbers in $\mathbf{Q}_p$}{}

\section{Introduction}

The \code{padic_t} data type represents elements of $\mathbf{Q}_p$ to 
precision~$N$, stored in the form $x = p^v u$ with $u, v \in \mathbf{Z}$. 
Arithmetic operations can be carried out with respect to a context 
containing the prime number $p$ and various pieces of pre-computed data.

Independent of the context, we consider a $p$-adic number 
$x = u p^v$ to be in \emph{canonical form} whenever either 
$p \nmid u$ or $u = v = 0$, and we say it is \emph{reduced} 
if, in addition, for non-zero~$u$, $u \in (0, p^{N-v})$.

We briefly describe the interface:

The functions in this module expect arguments of type \code{padic_t}, 
and each variable carries its own precision.  The functions have an 
interface that is similar to the MPFR functions.  In particular, they 
have the same semantics, specified as follows:  Compute the requested 
operation exactly and then reduce the result to the precision of the 
output variable.

\input{input/padic.tex}

%%%%%%%%%%%%%%%%%%%%%%%%%%%%%%%%%%%%%%%%%%%%%%%%%%%%%%%%%%%%%%%%%%%%%%%%%%%%%%%%
% Arithmetic functions                                                         %
%%%%%%%%%%%%%%%%%%%%%%%%%%%%%%%%%%%%%%%%%%%%%%%%%%%%%%%%%%%%%%%%%%%%%%%%%%%%%%%%

\chapter{arith}
\epigraph{Arithmetic functions}{}

\section{Introduction}

This module implements arithmetic functions, number-theoretic and 
combinatorial special number sequences and polynomials.

\input{input/arith.tex}

%%%%%%%%%%%%%%%%%%%%%%%%%%%%%%%%%%%%%%%%%%%%%%%%%%%%%%%%%%%%%%%%%%%%%%%%%%%%%%%%
% Unsigned single limb arithmetic                                              %
%%%%%%%%%%%%%%%%%%%%%%%%%%%%%%%%%%%%%%%%%%%%%%%%%%%%%%%%%%%%%%%%%%%%%%%%%%%%%%%%

\chapter{ulong\_extras}
\epigraph{Unsigned single limb arithmetic}{}

\section{Introduction}

This module implements functions for single limb unsigned integers, 
including arithmetic with a precomputed inverse and modular arithmetic.

The module includes functions for square roots, factorisation and 
primality testing. Almost all the functions in this module are highly
developed and extremely well optimised.

The basic type is the \code{mp_limb_t} as defined by MPIR. Functions
which take a precomputed inverse either have the suffix \code{preinv} 
and take an \code{mp_limb_t} precomputed inverse as computed by 
\code{n_preinvert_limb} or have the suffix \code{_precomp} and accept
a \code{double} precomputed inverse as computed by 
\code{n_precompute_inverse}. 

Sometimes three functions with similar names are provided for the 
same task, e.g. \code{n_mod_precomp}, \code{n_mod2_precomp} and
\code{n_mod2_preinv}. If the part of the name that designates the
functionality ends in 2 then the function has few if any limitations 
on its inputs. Otherwise the function may have limitations such as
being limited to 52 or 53 bits. In practice we found that the 
preinv functions are generally faster anyway, so most times it pays
to just use the \code{n_blah2_preinv} variants.

Some functions with the \code{n_ll_} or \code{n_lll_} prefix accept
parameters of two or three limbs respectively.

\section{Simple example}

The following example computes $ab \pmod{n}$ using a precomputed 
inverse, where $a = 12345678, b = 87654321$ and $n = 111111111$.

\begin{lstlisting}[language=c]
#include <stdio.h>
#include "ulong_extras.h"
...
mp_limb_t r, a, b, n, ninv;

a = 12345678UL;
b = 87654321UL;
n = 111111111UL;
ninv = n_preinvert_limb(n);

r = n_mulmod2_preinv(a, b, n, ninv);

printf("%lu*%lu mod %lu is %lu\n", a, b, n, r);
\end{lstlisting}

The output is:
\begin{lstlisting}[language=c]
12345678*87654321 mod 111111111 is 23456790
\end{lstlisting}

\input{input/ulong_extras.tex}

%%%%%%%%%%%%%%%%%%%%%%%%%%%%%%%%%%%%%%%%%%%%%%%%%%%%%%%%%%%%%%%%%%%%%%%%%%%%%%%%
% Signed single limb arithmetic                                                %
%%%%%%%%%%%%%%%%%%%%%%%%%%%%%%%%%%%%%%%%%%%%%%%%%%%%%%%%%%%%%%%%%%%%%%%%%%%%%%%%

\chapter{long\_extras}
\epigraph{Signed single limb arithmetic}{}

\input{input/long_extras.tex}

%%%%%%%%%%%%%%%%%%%%%%%%%%%%%%%%%%%%%%%%%%%%%%%%%%%%%%%%%%%%%%%%%%%%%%%%%%%%%%%%
% Fast Fourier Transforms                                                      %
%%%%%%%%%%%%%%%%%%%%%%%%%%%%%%%%%%%%%%%%%%%%%%%%%%%%%%%%%%%%%%%%%%%%%%%%%%%%%%%%

\chapter{fft}
\epigraph{Fast Fourier Transforms}{}

\input{input/fft.tex}

%%%%%%%%%%%%%%%%%%%%%%%%%%%%%%%%%%%%%%%%%%%%%%%%%%%%%%%%%%%%%%%%%%%%%%%%%%%%%%%%
% Quadratic Sieve                                                              %
%%%%%%%%%%%%%%%%%%%%%%%%%%%%%%%%%%%%%%%%%%%%%%%%%%%%%%%%%%%%%%%%%%%%%%%%%%%%%%%%

\chapter{qsieve}
\epigraph{Quadratic sieve}{}

\input{input/qsieve.tex}

%%%%%%%%%%%%%%%%%%%%%%%%%%%%%%%%%%%%%%%%%%%%%%%%%%%%%%%%%%%%%%%%%%%%%%%%%%%%%%%%
% longlong.h                                                                   %
%%%%%%%%%%%%%%%%%%%%%%%%%%%%%%%%%%%%%%%%%%%%%%%%%%%%%%%%%%%%%%%%%%%%%%%%%%%%%%%%

\chapter{longlong.h}
\epigraph{$64$-bit arithmetic}

\input{input/longlong.tex}

%%%%%%%%%%%%%%%%%%%%%%%%%%%%%%%%%%%%%%%%%%%%%%%%%%%%%%%%%%%%%%%%%%%%%%%%%%%%%%%%
% mpn_extras                                                                   %
%%%%%%%%%%%%%%%%%%%%%%%%%%%%%%%%%%%%%%%%%%%%%%%%%%%%%%%%%%%%%%%%%%%%%%%%%%%%%%%%

\chapter{mpn\_extras}

\input{input/mpn_extras.tex}

%%%%%%%%%%%%%%%%%%%%%%%%%%%%%%%%%%%%%%%%%%%%%%%%%%%%%%%%%%%%%%%%%%%%%%%%%%%%%%%%
% C++ wrapper                                                                  %
%%%%%%%%%%%%%%%%%%%%%%%%%%%%%%%%%%%%%%%%%%%%%%%%%%%%%%%%%%%%%%%%%%%%%%%%%%%%%%%%

\chapter{flintxx}
\epigraph{C++ wrapper library for FLINT}

\section{Introduction}

FLINT provides fast algorithms for number theoretic computations. For many
reasons, it is written in C. Nonetheless, some users prefer object oriented
syntax. FLINT ships with a set of wrapper C++ classes, together termed flintxx,
which provide such an object oriented syntax.

In this chapter, we describe how to \emph{use} flintxx. The FLINT developer
wishing to extend the wrapper library should consult the
appendix~\ref{app:genericxx}.

In general, flintxx strives to behave just like the underlying FLINT C
interface, except that we use classes and C++ operators to make the client code
look more pleasant. The simple example from the section on \code{fmpz} can be
transcribed into C++ as follows:

\begin{lstlisting}[language=c++]
#include "fmpzxx.h"
...
using namespace flint;
fmpzxx x, y;
x = 7u;
y = x*x;
std::cout << x << "^2 = " << y << std::endl;
\end{lstlisting}

As can be seen here, and in general, if a FLINT C interface is called \code{foo}
and resides in \code{foo.h}, then the C++ version is called \code{fooxx} and
resides in \code{fooxx.h}. All flintxx classes live inside
\code{namespace flint}.

\section{Overview}

\paragraph{Expression templates}
The implementation of flintxx tries very hard not to incur any overhead over
using the native C interface. For this reason, we use \emph{expression
templates} for lazily evaluating expressions. This allows us to avoid creating
excessively many temporaries, for example. This means that even if \code{x} and
\code{y} are of type \code{fmpzxx} (say), then \code{x + y} will \emph{not} be
of type \code{fmpzxx}. Instead it will be an object which for most purposes
behaves just like \code{fmpzxx}, but really only expresses the fact that it
represents the sum of \code{x} and \code{y}.

This distinction almost never matters, since expression templates are evaluated
automatically in most cases. Thus \code{cout << x + y} or \code{x + y == 7} will
work just as one might expect. There are ways to request explicit evaluation of
an expression template, most notably \code{(x + y).evaluate()} and
\code{fmpzxx(x + y)}.

One caveat of the expression template approach is that compiler error messages
can be long and hard to understand. TODO static assert

\paragraph{Reference types}

One subtlety in wrapping a C library is that references do not work as easily as
one might expect. If we were to only ever use the \code{fmpzxx} class, and never
use \code{fmpz_t}, there would be no problem in just using \code{fmpzxx&} and
\code{const fmpzxx&} as reference types. However, in this case there is no way
of turning an \code{fmpz_t} into an object which can interact with flintxx,
without copying the entire \code{fmpz}. This may not seem like a big problem,
until one realises that (e.g.) \code{fmpq_t} internally consists of two
instances of \code{fmpz_t}, and that \code{fmpqxx} should be able to expose
numerator and denominator.

For this reason, for every C interface \code{foo}, flintxx provides two
additional types, called \code{fooxx_ref} and \code{fooxx_srcref}, acting as
replacements for \code{fooxx&} and \code{const foox&}, respectively. Instances
of \code{fooxx_ref} or \code{fooxx_srcref} behave almost exactly like instances
\code{fooxx}, in fact the user should never notice a difference. Any flintxx
operation or expression which works on objects of type \code{foo} also works on
objects of type \code{fooxx_ref} and \code{fooxx_srcref}. Moreover, instances
of \code{foo} can be converted implicitly to \code{fooxx_ref} or
\code{fooxx_srcref}, and \code{fooxx_ref} can be converted implicitly to
\code{fooxx_srcref}. It is also possible to explicitly convert reference types
\code{fooxx_*ref} to \code{fooxx} (since this entails copying, we provide no
implicit conversion).

\paragraph{Extending flintxx}

Explanations on the inner workings of the flintxx expression template library
and how they pertain to wrapping new C interfaces can be found in
appendix~\ref{app:genericxx}. Here we just want to remark that the flintxx
classes are not designed for inheritance. Instead, if you want to modify
slightly the behaviour of one of the flintxx classes for your purposes, use
composition: TODO describe forwarding.h.

\section{Notations and conventions for the C++ interface documentation}
As explained above, the flintxx classes and functions perform quite a number of
operations which should be invisible to the user. Some template types implement
methods which only make sense for some template arguments, etc. In what follows,
we document a ``virtual'' set of classes and functions, which explain how the
user should expect its objects to behave, and which we guarantee to maintain.
Other interfaces should be considered implementation details and subject to
change.

For any interface \code{foo}, we introduce virtual types \code{Foo_expr},
\code{Foo_target} and \code{Foo_src}. We use type names beginning with capital
letters to denote the virtual types. The virtual type \code{Foo_expr} represents
any
expression built out of \code{fooxx}, \code{fooxx_ref} and \code{fooxx_srcref}
by overloaded operators, lazily evaluated functions etc. The type
\code{Foo_target} represents either \code{fooxx&} or \code{fooxx_ref}, whereas
\code{Foo_src} represents one of \code{fooxx}, \code{fooxx_ref} or
\code{fooxx_srcref}. For example, the static method documented as
\code{fmpzxx::randm} only exists (nominally) in the class \code{fmpzxx}, and not
in a reference type or expression template. On the other hand the method
documented as \code{Fmpz_expr::is_zero} works on any expression built from
fmpzxx or its reference types.

When using virtual types, we will suppress reference notation. No flintxx types
are ever copied automatically, unless the documentation explicitly says so.

It is also often the case that flintxx functions are conditionally enabled
templates. A notation such as \code{void foo(T:is_signed_integer)} denotes a
template function which is enabled whenever the template parameter \code{T}
satisfies the \emph{type trait} \code{is_signed_integer}. These type traits
should be self-explanatory, and in any case are documented in TODO.

In what follows, we will never document copy constructors, or
or implicit conversion constructors pertaining to reference types. We will also
not document assignment operators for expressions of the same type. (Thus if
\code{x} is an \code{fmpzxx} and \code{y} is an \code{fmpqxx}, then \code{x = x}
and \code{y = x} are both valid, but only the second assignment operator is
documented explicitly.)

Most flintxx functions and methods wrap underlying C functions in a way which is
evident from the signature of the flintxx function/method. If this is the case,
no further documentation is provided. For example, the function
\code{double dlog(Fmpz_expr x)} simply wraps
\code{double fmpz_dlog(const fmpz_t)}.  As is
evident from the return type, \code{dlog} immediately evaluates its argument,
and then computes the logarithm. In contrast, a function like
\code{Fmpz_expr gcd(Fmpz_expr, Fmpz_expr)} returns a lazily evaluated expression
template (and wraps \code{void fmpz_gcd(fmpz_t, const fmpz_t, const fmpz_t)}).

\input{input/flintxx.tex}

%%%%%%%%%%%%%%%%%%%%%%%%%%%%%%%%%%%%%%%%%%%%%%%%%%%%%%%%%%%%%%%%%%%%%%%%%%%%%%%%
% profiler                                                                     %
%%%%%%%%%%%%%%%%%%%%%%%%%%%%%%%%%%%%%%%%%%%%%%%%%%%%%%%%%%%%%%%%%%%%%%%%%%%%%%%%

\chapter{profiler}

\input{input/profiler.tex}

%%%%%%%%%%%%%%%%%%%%%%%%%%%%%%%%%%%%%%%%%%%%%%%%%%%%%%%%%%%%%%%%%%%%%%%%%%%%%%%%
% interfaces                                                                   %
%%%%%%%%%%%%%%%%%%%%%%%%%%%%%%%%%%%%%%%%%%%%%%%%%%%%%%%%%%%%%%%%%%%%%%%%%%%%%%%%

\chapter{interfaces}
\epigraph{Interfaces to other packages}{}

\section{Introduction}

In this chapter we provide interfaces to various external packages.

\input{input/interfaces.tex}

%%%%%%%%%%%%%%%%%%%%%%%%%%%%%%%%%%%%%%%%%%%%%%%%%%%%%%%%%%%%%%%%%%%%%%%%%%%%%%%%
% extending the C++ wrapper                                                    %
%%%%%%%%%%%%%%%%%%%%%%%%%%%%%%%%%%%%%%%%%%%%%%%%%%%%%%%%%%%%%%%%%%%%%%%%%%%%%%%%
\appendix
\chapter{Extending the C++ wrapper}
\label{app:genericxx}

\section{Introduction}

\input{input/genericxx.tex}

%%%%%%%%%%%%%%%%%%%%%%%%%%%%%%%%%%%%%%%%%%%%%%%%%%%%%%%%%%%%%%%%%%%%%%%%%%%%%%%
% BACKMATTER                                                                  %
%%%%%%%%%%%%%%%%%%%%%%%%%%%%%%%%%%%%%%%%%%%%%%%%%%%%%%%%%%%%%%%%%%%%%%%%%%%%%%%

\backmatter

\phantomsection
\addcontentsline{toc}{chapter}{References}
\bibliographystyle{amsplain}
\bibliography{flint-manual}

\end{document}
